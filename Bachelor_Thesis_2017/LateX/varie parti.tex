In meccanica classica date le equazioni di Newton e il principio di unicità di Cauchy, si può descrivere univocamente l' orbita descritta dal sistema preso in esame.
In generale queste equazioni lasciano definita quest'orbita tramite delle equazioni differenziali del secondo ordine che non sempre sono integrabili.
Consideriamo, infatti un modello elementare di \textit{n} corpi in interazione solo attraverso la forza gravitazionale. Ad oggi, solo per $\textit{n}=2$ troviamo un soluzione esplicita del problema. 




\section { Discussione riguardo al metodo analitico usato }

Un'attenta analisi critica della metodologia impegata rivela due possibili miglioramenti per implementare gli strumenti utilizzati:
\\i) partendo dalla coordinate polari, si può trovare una famiglia di coordinate canoniche che descrivono meglio il problema;
\\ii) la norma utilizzata 


Per cominciare si definiscono gli elementi caratteristici dei corpi del problema.\\
Siano $(P,Q)$ , $(p,q)$, $(p',q')$ le coordinate rispettivamente del Sole, Giove e di un preciso asteroide. In più, sia M la massa del Sole, m quella di Giove e $\epsilon$ quella dell'asteroide. Denotiamo, infine, con $a$ la distanza Sole-Giove, ovvero $a:=||Q-q||$.
\\Dunque, l'hamiltoniana del problema è $$ H = H_0 + H' $$ \\dove $H_0$ è l'hamiltonia dei due "primari", mentre 
$$H' = \frac{{p'}^2}{2 \epsilon} - \frac{\epsilon G M}{||q' - Q||} - \frac{\epsilon G m}{||q' - q||} $$.
Nel limite in cui la massa $\epsilon$ tende a zero, con evidenza $H'$\footnote {L'hamiltoniana $H'$ può essere resa independente da $\epsilon$ con una trasformazione canonica.} descriverà l'hamiltoniana del terzo corpo e $Q(t)$, $q(t)$ saranno le soluzioni del problema circolare a due corpi dato da $H_0$.
\\\\Assunto, inoltre, che il moto dell'asteroide si svolga sul piano dei primari e che il moto di quest'ultimi sia circolare, si applicano in serie due cambi di coordinate. 
In primo luogo, si passa a coordinate baricentriche con l'asse delle ascisse posto sulla direttrice dei due pianeti orientato verso il Sole. Successivamente, si considera un sistema di riferimento corotante con i primari con velocità $\omega$\footnote{$\omega \in \mathbb{R}^3$ t.c. $\omega^2*a^3 = G*(M+m)$}. 
Si scelgono, in aggiunta, unità di misura tali che, definita $$\mu := \frac{m}{m+M}  $$, \\$a=1$, la velocità di rivoluzione di Giove $\omega$ sia unitaria, la massa del Sole sia $ 1-\mu $ e quella di Giove $ \mu $. Complessivamente, quindi, il sistema avrà massa 1 e il Sole starà costantemente sul punto $ Q*= (\mu,0,0)$ mentre Giove in $q*= (1-\mu,0,0)$.
\\\\Rimosso ogni orpello formale, l'hamiltonia trasformata diventa \begin{equation}
\label{hamiltonianaRTBP}
H(q,p)= \frac{p^2}{2} -(q_1p_2 - q_2p_1) - \frac{1-\mu}{R} - \frac{\mu}{r}
\end{equation}
\\ove $R:=||q-Q*||$,$r:=||q-q*||$

\subsection {Punti di equilibrio lagrangiani}

Ponendo uguale a zero ogni derivata parziale della \ref{hamiltonianaRTBP} si ha
$$p_3=0, \quad q_3=0,\quad p_1=-q_2,\quad p_2=q_1$$, 
\begin{equation} 
\label{eq:Plagr1}
q_2(-1+ \frac{(1-\mu)}{R^{3/2}} + \frac{\mu}{r^{3/2}} )  
\end{equation}
\begin{equation} 
\label{eq:Plagr2}
-q_1 + \frac{(1-\mu)(q_1+\mu)}{R^{3/2}} + \frac{\mu (q_1-q+\mu)}{r^{3/2}} )
\end{equation}
\\\\Risolvendo le \ref{eq:Plagr1} e \ref{eq:Plagr2} si ottengo i punti d'equilibrio o punti "lagrangiani": posto $q_2=0$ nella \ref{eq:Plagr2} otteniamo $L_1$,$L_2$,$L_3$ detti "collineari"; mentre con $q_2\neq0$ e $R=r$ si determinano $L4$ e $L5$ detti "triangolari". Quest'ultimi, infatti, giacciono rispettivamente sul vertice superiore e inferiore dei due triangoli equilateri formati con i primari.


\section {Natura di L4-L5}

Individuati i punti lagrangiani, si trasla l'hamiltoniana su L4 in modo da poter applicare la teoria delle piccole oscillazioni. In particolare, si linearizzerà l'hamiltoniana nella nuova origine e, ottenuti i suoi autovalori, si discuterà la natura dell'equilibrio.
Formalmente essendo $L4$ dato da $ q_1 = -\frac{(1- 2\mu)}{2} $, $ q_2 = \frac{\sqrt{3}}{2} $, $ p_1 = -q_2 $, $ p_2 =q_1 $ e sviluppata \ref{hamiltonianaRTBP} attorno a questo punto, si trova che $$\tilde{H} = \tilde{H_2} + O(\tilde{q}^3) $$ \\dove le hamiltoniane "tildate" sono espresse nelle nuove coordinate $(\tilde{p},\tilde{q})$.
\\Nel dettaglio, $H_2 = \mathcal{H}(p_1,p_2,q_1,q_2) + \frac{1}{2} (p_3^2+q_3^2)$ il cui grado di libertà $(p_3,q_3)$ è rappresentato da un oscillatore armonico disaccoppiato dagli altri due e che precede con $\omega_3=1$, pari a quella dei primari.
\\D'altra parte, le equazioni del moto per gli altri due gradi di libertà si ricavano facilmente.
Posto $x:=(p_1,p_2,q_1,q_2)$, le equazioni avranno la forma $$\dot{x}=Ax, \quad
A:= \left( \begin{array}{llcl}
0&1&a&b \\ -1&0&b&d \\ 1&0&0&1 \\ 0&1&-1&0

\end{array}\right)  $$ \\ove $a= -1/4$, $b = \frac{3 \sqrt3}{4} (1-2\mu)$, $d= 5/4$.

Notando che $H_2$ è reale simplettica, si sa che per ogni autovalore $\lambda$ ci saranno anche gli autovalori $-\lambda$, $\bar{\lambda}$, $-\bar{\lambda}$.
In più, si nota che l'equazione agli autovalori è biquadratica. Dunque, per coppie di autovalori immaginari puri, il punto sarà stabile, o anche detto "ellittico"; mentre se almeno un autovalore avrà parte reale negativa, il punto sarà instabile, o "iperbolico".
Nel dettaglio, si trova che, definito $$ \mu_R = \frac{1}{2} ( 1- \sqrt{\frac{23}{27}}) \simeq 0.0385 $$  ("limite di Routh"): 
\begin{itemize}
	\item se $\mu \leq \mu_R \quad (1-\mu) \leq \mu_R$ allora l'equilibrio è ellittico, in quanto troviamo solo autovalori immaginari puri;
	\item altrimenti è iperbolico.

In questo contesto scelta la massa del Giove $m= 1.898,19*10^{24} Kg$, la massa di Sole $M=1.988.500*10^{24}$ [NASA], $\mu_R \simeq 0.001$. 

\subsection{Problemi di stabilità} 

Per $\mu \leq \mu_R$ si può passare all'hamiltoniana $K(P,Q)$ espressa in coordinate normali $P_1,...,Q_3$ che rendono il sistema disaccoppiato in tre oscillatori armonici.
$$ K(P,Q)= \frac{\omega_1}{2} ( Q_1^2+P_1^2) - \frac{\omega_2}{2} ( Q_2^2+P_2^2) + \frac{\omega_3}{2} ( Q_3^2+P_3^2)$$
\\ove $\omega_1=\sqrt{\frac{1+\alpha}{2}}$, $\omega_2=\sqrt{\frac{1-\alpha}{2}}$, $\omega_3=1$.
Dato che $\omega_2 < 0$, l'origine non è più un punto di minimo assoluto per l'hamiltoniana $K$ del sistema completo (precisamente L4 è punto di sella). Quindi, a differenza del sistema linearizzato, l'aggiunta di termini di $O(q^3)$ saranno fondamentali per determinare la natura del punto sotto la dinamica di \ref{hamiltonianaRTBP}.



PROVA:

$$ e_j + if_j= \left( \begin{array}{llcl}
\frac{8 \omega_j^2 + 4 \sqrt{3} \alpha +9}{8} \\ \frac{16i \omega_j+4 \alpha +3 \sqrt{3}}{8} \\ i \omega_j \( \frac{8 \omega_j^2 + 4 \sqrt{3} \alpha}{8}\) 
\\ i \omega_j \frac{4 \alpha +3\sqrt{3}}{8} + \frac{4\alpha\sqrt{3}+9}{4}
\end{array}\right)$$

con costanti $m_j (j=1,2)$ date dalla $m_j=\omega_j D_j$, 
$$ D_j = \( \frac{8\omega_j^2 + 4 \sqrt{3}\alpha}{8}\)^2 - 2\(\sqrt{3}\alpha + \frac{9}{4} \) + \(\frac{4\alpha+3\sqrt{3}}{8}\)^2$$ \\ e $\omega_1^2$, $\omega_2^2$, $\alpha$ sono definite come $$ \omega_1^2 = \frac{1}{2} + \frac{1}{2} \sqrt{1 - \frac{27}{4} + 4\alpha^2} \quad  \omega_2^2 = \frac{1}{2} - \frac{1}{2} \sqrt{1 - \frac{27}{4} + 4\alpha^2} \quad \alpha - \frac{(1-2\mu)e\sqrt{3}}{4} $$